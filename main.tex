%%%%%%%%%%%%%%%%%%%%%%%%%%%%%%%%%%%%%%%%%
%  My documentation report
%  Objetive: Explain what I did and how, so someone can continue with the investigation
%
% Important note:
% Chapter heading images should have a 2:1 width:height ratio,
% e.g. 920px width and 460px height.
%
%%%%%%%%%%%%%%%%%%%%%%%%%%%%%%%%%%%%%%%%%


%----------------------------------------------------------------------------------------
%	PACKAGES AND OTHER DOCUMENT CONFIGURATIONS ----------------------------------------------------------------------------------------

\documentclass[12pt,fleqn]{book} % Default font size and left-justified equations

\usepackage[top=3cm,bottom=3cm,left=3.2cm,right=3.2cm,headsep=10pt,letterpaper]{geometry} % Page margins

\usepackage{xcolor} % Required for specifying colors by name
\definecolor{ocre}{RGB}{52,177,201} % Define the orange color used for highlighting throughout the book

% Font Settings
\usepackage{avant} % Use the Avantgarde font for headings
%\usepackage{times} % Use the Times font for headings
\usepackage{mathptmx} % Use the Adobe Times Roman as the default text font together with math symbols from the Sym\u00adbol, Chancery and Com\u00adputer Modern fonts
\usepackage{microtype} % Slightly tweak font spacing for aesthetics
\usepackage[utf8]{inputenc} % Required for including letters with accents
\usepackage[T1]{fontenc} % Use 8-bit encoding that has 256 glyphs
\usepackage{amsthm}

% Bibliography
\usepackage[style=alphabetic,sorting=nyt,sortcites=true,autopunct=true,babel=hyphen,hyperref=true,abbreviate=false,backref=true,backend=biber]{biblatex}
\addbibresource{bibliography.bib} % BibTeX bibliography file
\defbibheading{bibempty}{}

\input{structure} % Insert the commands.tex file which contains the majority of the structure behind the template

%----------------------------------------------------------------------------------------
%	Definitions of new commands
%----------------------------------------------------------------------------------------

\def\R{\mathbb{R}}
\newcommand{\cvx}{convex}
\begin{document}

%----------------------------------------------------------------------------------------
%	TITLE PAGE
%----------------------------------------------------------------------------------------

\begingroup
\thispagestyle{empty}
\AddToShipoutPicture*{\put(0,0){\includegraphics[scale=1]{ib.jpg}}} % Image background
\centering
\vspace*{5cm}
\par\normalfont\fontsize{35}{35}\sffamily\selectfont
\textbf{Computer Science Bible}\\
{\LARGE IB.CS-HL 1 \& 2}\par % Book title
\vspace*{1cm}
{\Huge Nerds, et al}\par % Author name
\endgroup

%----------------------------------------------------------------------------------------
%	COPYRIGHT PAGE
%----------------------------------------------------------------------------------------

\newpage
~\vfill
\thispagestyle{empty}

%\noindent Copyright \copyright\ 2014 Andrea Hidalgo\\ % Copyright notice

\noindent \textsc{The Village School at Houston, Texas}\\

\noindent {https://github.com/bishan-batel/ibcs-bible}\\ % URL

\noindent Research done by the IB Moderation bullshit team \\ % License information

\noindent \textit{First release, August 1422} % Printing/edition date

%------------------------------------------------------------------------------
%	TABLE OF CONTENTS
%------------------------------------------------------------------------------
\chapterimage{csbs1.jpg} % Table of contents heading image

\pagestyle{empty} % No headers

\tableofcontents % Print the table of contents itself

%\cleardoublepage % Forces the first chapter to start on an odd page so it's on 
% the right

\pagestyle{fancy} % Print headers again

%------------------------------------------------------------------------------
%	CHAPTER 1
%------------------------------------------------------------------------------
\chapterimage{csbs1.jpg} % Chapter heading image
\chapter{Systems}

\section{Properties of Systems}

\begin{definition}[System]
    A combination of hardware and software that interact regularly to perform 
    all aspects of managing and processing information, especially within a 
    large organization.
\end{definition}\par

\subsection{Reasons for a new system}
\begin{itemize}
  \item To replace an existing system
  \item To improve an existing system
  \item To provide a new service
  \item To provide a new product
  \item To provide a new business
\end{itemize}

\section{Change Management}
\begin{definition}[Change Management]
  The process of handling change with the least amount of disruption to the
  organization.
\end{definition}\par

\subsection{Stages of Change Management}

% diagram of 6 circles in a wheel connected to each other
\begin{center}
\smartdiagram[circular diagram:clockwise]{
  Analyze,
  Decision,
  Agreement,
  Planning,
  Implementation,
  Assessment
}
\end{center}

\subsection{Considerations in Change Management}
\begin{itemize}
  \item Staff Training
  \item Data migration
  \item Switching over
  \item Recovery Software in case of a Disaster
  \item Help Systems
  \item Business Process
\end{itemize}

\subsection{System Life Cycle}
\begin{definition}[System Life Cycle]
  The stages that the development of a new system goes through.
\end{definition}\par

\begin{center}
  \smartdiagramset{
    text width=2.5cm,
    circular distance=5.7cm,
  }
  \smartdiagram[circular diagram:clockwise]{
    Definition,
    Analysis,
    Design,
    Implementation,
    Testing,
    Installation,
    Documentation,
    Evaluation,
    Maintenance
  }
\end{center}

\section{Stakeholders}
\begin{definition}[Stakeholder]
  Inidividuals who stand to gain or lose something from the success or failure
  of an existing or proposed system.
\end{definition}\par

\begin{example}
  Types of Stakeholders:
  \begin{itemize}
    \item System Owners
    \item System Users
    \item Project Managers
    \item External Service Provider
    \item Investors
  \end{itemize}
\end{example}

\begin{definition}[End User]
  The person or group who will use the product.
\end{definition}\par

\subsection{Role of End Users}
\begin{itemize}
  \item In planning stage you are able to identify problems
  \item Create simpler methods/systems
  \item User involvement leads to more reliable ways to organize features
  \item Enables system developers to know and understand user's lexicon, so developers can communicate using the same language
  \item Eliminates misunderstandings and reduces errors
  \item It can gain user agreement
\end{itemize}

\begin{definition}[Consequences of not involving end users in the design process]
  User frustration when using system, developer could create useless solution,
  company losses productivity.
\end{definition}\par

\begin{example}[Roles end users can have during the process of creating a 
  \emph{new} system]
  \phantom{-}

  \begin{itemize}
    \item Development (end users can tell developer what they want the progarm to do
    \item Can Help in the design process by telling the developer what they want to see in the program
    \item Testing (beta or user acceptance testing)
  \end{itemize}
\end{example}

\subsection{Methods of Obtaining Requirements from Stakeholders}
\begin{itemize}
  \item Interviews
  \item Direct Observation
  \item Surveys
\end{itemize}

\section{Range of Usability Problems with commonly used digital devices}

\subsection{Usability}
\begin{definition}[Effectiveness]
\end{definition}\par

\begin{definition}[Effiency]
\end{definition}\par

\begin{definition}[Ergonomics]
  In usability of system design, ergonomics is the qualities of system design 
  that makes it safe and comfortable to use.
\end{definition}\par
Some examples of ergonomics could be the size of the keyboard, the size of the 
screen, the resolution of the screen, the size of the mouse, the size of the 
buttons, 

\begin{align*}
\end{align*}

\begin{definition}[Accessibility]
\end{definition}\par


%	Digital Devices -----------------------------------------------------------------------------
\subsection{Digital Devices}

\end{document}
